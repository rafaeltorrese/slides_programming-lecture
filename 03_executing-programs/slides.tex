\documentclass[../main.tex]{subfiles}
\graphicspath{{figs/}}



\title{Programs, Algorithms, and Functions}
%\subtitle{OPERATIONS RESEARCH. \\ Mathematical Models} % 6 hours. 2 sessions
\AtBeginSection[] % Do nothing for \section* %
{
\begin{frame}<beamer> 
  \frametitle{Agenda}
  \tableofcontents[currentsection] 
\end{frame}
}

\begin{document}
% ==================================================
\begin{frame}
  \maketitle
\end{frame}
\begin{frame}{Agenda}
  \tableofcontents
\end{frame}
% ==================================================

\section{Python Scripts}
\label{sec:introduction}

\begin{frame}{Introduction}
  \begin{itemize} \parskip3mm \justifying
  \item   Our experience with computers is a machine with a huge volume of carefully organized
    logic.
\item No one piece of this logic is necessarily complex or can capture what drives the
result. 
\item Rather, the entire system is organized such that it comes together to provide the
output you expect
  \end{itemize}
\end{frame}

\begin{frame}{Python Scripts and Modules}
  \begin{itemize} \parskip3mm \justifying
  \item You may be aware that most Python code lives in text files with a \textbf{.py} extension. 
  \item These files are simply plain text and can be edited with any text editor. Programmers typically edit these files using either a text editor such
as Notepad++, or Integrated Development Environments (IDEs) such as Jupyter or PyCharm.
\item Typically, standalone \textbf{.py} files are either called \alert{scripts} or \alert{modules}. 
\item A script is a file that is designed to be executed, usually from the command line. 
\item A  module is usually imported into another part of the code or an interactive shell to be executed. 
  \end{itemize}
\end{frame}

\section{Python Algorithms}
\label{sec:algorithms}


\section{Basic Functions}
\label{sec:functions}



% ==================================================
\end{document}



