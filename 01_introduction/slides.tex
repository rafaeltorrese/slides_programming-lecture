\documentclass[../main.tex]{subfiles}
\graphicspath{{figs/}}



\title{Math, Strings, Conditionals and Loops}
%\subtitle{OPERATIONS RESEARCH. \\ Mathematical Models} % 6 hours. 2 sessions
\AtBeginSection[] % Do nothing for \section* %
{
\begin{frame}<beamer> 
  \frametitle{Agenda}
  \tableofcontents[currentsection] 
\end{frame}
}

\begin{document}


\begin{frame}
  \maketitle
\end{frame}


\begin{frame}{Agenda}
  \tableofcontents
\end{frame}



\section{String Interpolation}
\label{sec:string-interpolation}

\begin{frame}[fragile]{Comma Separators}
  \begin{itemize} \parskip3mm \justifying
  \item   When writing strings, you may want to include variables in the output.
  \item String interpolation includes the variable names as placeholders within the string. There are two standard methods for achieving string interpolation: \textbf{comma separators} and \textbf{format}.
  
  \end{itemize}

  \begin{code}{}
\begin{lstlisting}
      italian_greeting = 'Ciao'
      print('Should we greet people with', italian_greeting, 'in North Beach?')
  \end{lstlisting}
  \end{code}

\end{frame}

\begin{frame}[fragile]{Format}
  \begin{itemize} \parskip3mm \justifying
  \item With format, as with commas, Python types, ints, floats, and so on, are converted into strings upon execution
  \item There’s another way of combining strings: \alert{formatted string literals}, more commonly known as \textbf{f-strings}
  \end{itemize}

  \begin{code}{}
    \begin{lstlisting}
      owner = 'Lawrence Ferlinghetti'
      age = 100
      
      print(
         'The founder of City Lights Bookstore, {}, is now {} years old.'.format(owner, age)
      )
      
      print(f'The founder of City Lights Bookstore, {owner}, is now {age} years old.')
    \end{lstlisting}
  \end{code}
\end{frame}

\begin{frame}[fragile]{The \texttt{len( )} Function}
  \begin{itemize}
  \item There are many built-in functions that are particularly useful for strings. 
  \item One such function is \texttt{len()}, which is short for length. 
  \item The \texttt{len()} function determines the number of characters in a given string.

    \begin{code}{}
      \begin{lstlisting}
        arabic_greeting = 'Ahlan wa sahlan.'
        print(len(arabic_greeting))
      \end{lstlisting}
    \end{code}
  \end{itemize}
\end{frame}

\begin{frame}{String Methods}

  \begin{exercise}{}
    \begin{enumerate}
    \item Set a new variable, called \textbf{name}.
    \item Now, convert the name into lowercase letters using the \textbf{lower()} function.
    \item Now, capitalize the name using the capitalize() function.
    \item Convert the name into uppercase letters using upper().
    \item Finally, count the number of o instances in the word \textbf{'Corey'}.
    \end{enumerate}
  \end{exercise}

  You can find more about string methods in this link: \href{https://www.w3schools.com/python/python_ref_string.asp}{String Methods in Python}

\end{frame}

\begin{frame}{Casting}
  \begin{itemize} \justifying \parskip3mm
  \item   It's common for numbers to be expressed as strings when dealing with input and output. 
  \item Note that '5' and 5 are different types. 
  \item We can easily convert between numbers and strings using the appropriate type keywords. 
  \end{itemize}

  \begin{exercise}{}
    \begin{enumerate} \justifying
    \item Determine the type of '5'
    \item Now, add '5' and '7'.
    \item Convert the '5' string to an \textbf{int}.
    \item Add '5' and '7' by converting them to \textbf{int} first.
    \end{enumerate}
  \end{exercise}
\end{frame}

\begin{frame}[fragile]{The \texttt{input( )} Function}
  The \texttt{input( )} function is a built-in function that allows user input. It's a little different than what we have seen so far.

  \begin{exercise}{}
    \begin{enumerate} \justifying
    \item Ask a user for their name. Respond with an appropriate greeting
    \item Now, set a variable that will be equal to the \texttt{input( )} function
    \item Finally, select an appropriate output. Use \texttt{print( )}.
    \end{enumerate}
  \end{exercise}
\end{frame}

\begin{frameact}{Using the \texttt{input()} Function to Rate Your Day}
  \begin{enumerate} \justifying
  \item Display a question prompting the user to rate their day on a number scale of 1 to 10.
  \item Save the user's input as a variable.
  \item Display a statement to the user that includes the number.
  \end{enumerate}
\end{frameact}

\begin{frame}[fragile]{String Indexing and Slicing}
  \begin{onlyenv}<1>
    \begin{block}{Indexing} \justifying
    \begin{itemize} \justifying
    \item The characters of Python strings exist in specific locations; in other words, their order counts. 
    \item The index is a numerical representation of where each character is located. 
    \item The first \alert{character is at index 0}, the second character is at index 1; the third character is at index 2, and so on
    \end{itemize}
  \end{block}
  \end{onlyenv}

  \begin{onlyenv}<2>
      \begin{code}{}
    \begin{lstlisting}
      destination = 'San Francisco'
      destination[0]
      destination[1]
      destination[2]

      # negative index
      destination[-1]
      destination[-2]
    \end{lstlisting}
  \end{code}
\end{onlyenv}

\begin{onlyenv}<2>
  \begin{code}{}
    \begin{lstlisting}
      bridge = 'Golden Gate'
      bridge[6]
    \end{lstlisting}
  \end{code}
\end{onlyenv}

\begin{onlyenv}<3>
  \begin{block}{Slicing} \justifying
    A slice is \alert{a subset of a string or other element}. A slice could be the whole element or one character, but it's more commonly a group of adjoining characters.

    \begin{code}{}
      \begin{lstlisting}
        destination[4:11]
        destination[0:3]
        destination[:8]
        destination[-3:]
      \end{lstlisting}
    \end{code}
  \end{block}
\end{onlyenv}
\end{frame}

\begin{frame}[fragile]{Booleans and Conditionals}
\only<1>{  Booleans, named after George Boole, take the values of \textbf{True} or \textbf{False}. Although the idea behind Booleans is rather simple, they make programming immensely more powerful.}

  \begin{block}<only@1>{Booleans} \justifying
In Python, a Boolean class object is represented by the bool keyword and has a value of \textbf{True} or \textbf{False}.
\end{block}

\begin{onlyenv}<2>
  \begin{exercise}{}
  \begin{enumerate}
  \item Use a Boolean to classify someone as being over 18.
  \item Use a Boolean to classify someone as not being over 21 
  \end{enumerate}
\end{exercise}
\end{onlyenv}


  \begin{block}<only@3>{Logical Operators}
    Booleans may be combined with the \textbf{and}, \textbf{or}, and \textbf{not} logical operators.
  \end{block}


  \begin{onlyenv}<3>
    \begin{code}{}
    \begin{lstlisting}
      over_18, over_21 = True, False
      over_18 and over_21
      over_18 or over_21
      not over_18
      not over_21 or (over_21 or over_18)
    \end{lstlisting}
  \end{code}
  \end{onlyenv}
\end{frame}

\begin{frame}[fragile]{Comparison Operators}
  \begin{exercise}{}
    \begin{enumerate} \justifying
    \item Now, set age as equal to 20 and include a comparison operator to check whether age is less than 13.
    \item Check whether age is greater than or equal to 20 \textbf{and} less than or equal to 21.
    \item Check whether age is equivalent to 21.
    \item Check whether age is equivalent to 19.
    \item Is 6 equivalent to 6.0 in Python? 
    \end{enumerate}
  \end{exercise}
\end{frame}

\begin{frame}[fragile]{Comparing Strings}
  Python uses the convention of alphabetical order to make sense of these comparisons. Think of a dictionary: when comparing two words, the word that comes later in the
  dictionary is considered greater than the word that comes before
  \begin{exercise}{}
    \begin{enumerate}
    \item Let's compare single letters
    \item Now, let's compare 'New York' and 'San Francisco'
    \end{enumerate}
  \end{exercise}
\end{frame}

\begin{frame}[fragile]{Conditionals}
  \begin{itemize} \justifying
  \item   Conditionals are used when we want to express code based upon a set of circumstances or values. 
  \item Conditionals evaluate Boolean values or Boolean expressions, and they are usually preceded by \textbf{'if'}.
  \end{itemize}
  
  \begin{exercise}{}
    \begin{enumerate} \justifying
  \item Run multiple lines of code where you set the age variable to 20 and add an if clause
  \item Now, use nested conditionals 
    \end{enumerate}
  \end{exercise}
\end{frame}

\begin{frame}[fragile]{\texttt{if else}}
  
\only<1>{  \textbf{if} conditionals are commonly joined with else clauses. The idea is as follows. Say you want to print something to all users unless the user is under 18. You can address this with an \textbf{if-else} conditional.}

\begin{onlyenv}<2>
    \begin{exercise}{}
    \begin{enumerate} \justifying
    \item Introduce a voting program only to users over 18 by using the following code snippet
      \begin{lstlisting}
        age = 20
        if age < 18:
            print('You aren\'t old enough to vote.')
        else:
            print('Welcome to our voting program.')
      \end{lstlisting}
    \item Now run the following code snippet, which is an alternative to the code mentioned in step 1 of this exercise
    \end{enumerate}
\end{exercise}
\end{onlyenv}
\end{frame}

\begin{frame}[fragile]{The elif Statement}
  \begin{code}{}
    \begin{lstlisting}
      if age <= 10:
          print('Listen, learn, and have fun.')
      elif age<= 19:
          print('Go fearlessly forward.')
      elif age <= 29:
          print('Seize the day.')
      elif age <= 39:
          print('Go for what you want.')
      elif age <= 59:
          print('Stay physically fit and healthy.')
      else:
          print('Each day is magical.')
    \end{lstlisting}
  \end{code}
\end{frame}

\begin{frame}{Loops}
  There are three key components to most loops:
  
  \begin{enumerate} \justifying
  \item The start of the loop
  \item The end of the loop
  \item The increment between numbers in the loop
  \end{enumerate}
  
Python distinguishes between two fundamental kinds of loops: \textbf{while} loops, and \textbf{for} loops
\end{frame}

\begin{frame}[fragile]{The \texttt{while} Loops}
  In a while loop, a designated segment of code repeats provided that a particular condition is true. When the condition evaluates to false, the while loop stops running. The while loops print out the first 10 numbers.

  \begin{code}{}
    \begin{lstlisting}
      i = 1
      while i <= 10:
          print(i)
          i += 1
    \end{lstlisting}
  \end{code}

  \begin{code}{An Infinite Loop}
    \begin{lstlisting}
      x = 5
      while x <= 20:
          print(x)
    \end{lstlisting}
  \end{code}
\end{frame}

\begin{frame}[fragile]{\texttt{break} statement}
  \begin{itemize} \justifying
  \item \textbf{break} is a special keyword in Python that is specifically designed for loops. 
  \item If placed inside of a loop, commonly in a conditional, break will immediately terminate the loop. 
  \item It doesn't matter what comes before or after the loop. 
  \item The break is placed on its own line, and it breaks out of the loop.
  \end{itemize}

  To practice, you should print the first number greater than 100 that is divisible by 17
\end{frame}

\begin{frameact}{Finding the Least Common Multiple (LCM)}
  \begin{enumerate}\parskip3mm  \justifying
  \item Set a pair of variables as equal to 24 and 36.
  \item Initialize the while loop, based on a Boolean that is True by default, with an iterator.
  \item Set up a conditional to check whether the iterator divides both numbers.
  \item Break the while loop when the LCM is found.
  \item Increment the iterator at the end of the loop.
  \item  Print the results.
  \end{enumerate}
\end{frameact}

\end{document}
