\documentclass[../main.tex]{subfiles}
\graphicspath{{figs/}}



\title{Math, Strings, Conditionals and Loops}
%\subtitle{OPERATIONS RESEARCH. \\ Mathematical Models} % 6 hours. 2 sessions
\AtBeginSection[] % Do nothing for \section* %
{
\begin{frame}<beamer> 
  \frametitle{Agenda}
  \tableofcontents[currentsection] 
\end{frame}
}

\begin{document}


\begin{frame}
  \maketitle
\end{frame}


\begin{frame}{Agenda}
  \tableofcontents
\end{frame}



\section{String Interpolation}
\label{sec:string-interpolation}

\begin{frame}[fragile]{Comma Separators}
  \begin{itemize} \parskip3mm \justifying
  \item   When writing strings, you may want to include variables in the output.
  \item String interpolation includes the variable names as placeholders within the string. There are two standard methods for achieving string interpolation: \textbf{comma separators} and \textbf{format}.
  \item There’s another way of combining strings: \alert{formatted string literals}, more commonly known as \textbf{f-strings}
  \end{itemize}

\begin{lstlisting}
  italian_greeting = 'Ciao'
  print('Should we greet people with', italian_greeting, 'in North Beach?')
\end{lstlisting}
\end{frame}
\end{document}
