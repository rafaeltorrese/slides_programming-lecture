\documentclass[../main.tex]{subfiles}
\graphicspath{{figs/}}



\title{Python Structures}
%\subtitle{OPERATIONS RESEARCH. \\ Mathematical Models} % 6 hours. 2 sessions
\AtBeginSection[] % Do nothing for \section* %
{
\begin{frame}<beamer> 
  \frametitle{Agenda}
  \tableofcontents[currentsection] 
\end{frame}
}

\begin{document}
% ==================================================
\begin{frame}
  \maketitle
\end{frame}
\begin{frame}{Agenda}
  \tableofcontents
\end{frame}
% ==================================================
\begin{frame}[fragile]{Introduction}
  \begin{itemize} \justifying
  \item   In programming languages, data structures refer to objects that can hold some data together, which means they are used to store a collection of related data.
  \item For instance, you can use a list to store our to-do items for the day. 
  \item The following is an example to show you how lists are coded:
    \begin{lstlisting}
      todo = ["pick up laundry", "buy Groceries", "pay electric bills"]
    \end{lstlisting}
  \item We can also use a dictionary object to store more complex information such as subscribers' details from our mailing list
    \begin{lstlisting}
      User = {
        "first_name": "Jack",
        "last_name":"White",
        "age": 41,
        "email": "jack.white@gmail.com"
     }
    \end{lstlisting}
  \item There are four types of data structures in Python: \alert{list}, \alert{tuple}, \alert{dictionary}, and \alert{set}.
  \end{itemize}
\end{frame}

\section{Lists}
\label{sec:lists}

\begin{frame}[fragile]{The Power of Lists}
  \begin{enumerate} \justifying
  \item A list is a type of container in Python that is used to store \alert{multiple data sets at the same time}. 
  \item Python lists are often compared to arrays in other programming languages, but they do a lot more.
  \item A list in Python is written within square brackets, \texttt{[ ]}. Each element in the list has its own distinct position and index. 
  \end{enumerate}

  \begin{exercise}{Working with Python Lists}
      \begin{lstlisting}
        shopping = ["bread", "milk", "eggs"]
        print(shopping)
        
        for item in shopping:
            print(item)

        mixed = [365, "days", True]
        print(mixed)
        
      \end{lstlisting}
  \end{exercise}
\end{frame}

\begin{frame}[fragile]{Matrices as Nested Lists}
  \begin{exercise}{Using a Nested List to Store Data from a Matrix}
    \begin{lstlisting}
      m = [[1, 2, 3], [4, 5, 6]]
      print(m[1][1])

      for i in range(len(m)):
          for j in range(len(m[i])):
              print(m[i][j])

      for row in m:
          for col in row:
              print(col)
    \end{lstlisting}
  \end{exercise}
\end{frame}
\begin{frameact}{Using a Nested List to Store Employee Data}

  {\centering
    \begin{tabular}{lll}
      \toprule
      Name          & Age & Department \\
      \midrule
      John Mckee    & 38  & Sales      \\
      Lisa Crawford & 29  & Marketing  \\
      Sujan Patel   & 33  & HR\\
      \bottomrule
    \end{tabular}
    \par}
  
  \begin{enumerate}
\item Create a list and assign it to \textbf{employees}.
\item Create three nested lists in employees to store the information of each employee, respectively.
\item Print the employees variable.
\item Print the details of all employees in a presentable format.
\item Print only the details of Lisa Crawford.
  \end{enumerate}
\end{frameact}

\begin{frame}[fragile]{Matrix Operations}
  \begin{exercise}{Implementing Matrix Operations (Addition and Subtraction)}
    \begin{lstlisting}
      X = [[1,2,3],[4,5,6],[7,8,9]]
      Y = [[10,11,12],[13,14,15],[16,17,18]]

      result = [
      [0,0,0],
      [0,0,0],
      [0,0,0],
      ]

      # iterate through rows
      for i in range(len(X)):      
          # iterate through columns
          for j in range(len(X[0])):
              result[i][j] = X[i][j] + Y[i][j]
      print(result)
    \end{lstlisting}
  \end{exercise}
\end{frame}

\begin{frame}[fragile]{Matrix Multiplication Operations}
  \begin{exercise}{}
    \begin{lstlisting}
      X = [[1, 2], [4, 5], [3, 6]]
      Y = [[1,2,3,4],[5,6,7,8]]
      result = [[0, 0, 0, 0], [0, 0, 0, 0], [0, 0, 0, 0]]

      # iterating by row of X
      for i in range(len(X)):
          # iterating by column by Y
          for j in range(len(Y[0])):
              # iterating by rows of Y
              for k in range(len(Y)):
                  result[i][j] += X[i][k] * Y[k][j]
      print(result) 
    \end{lstlisting}
  \end{exercise}
\end{frame}

\begin{frame}{List Methods}
  \begin{itemize}\parskip3mm \justifying
  \item   Lists are one of the best data structures to use. 
  \item Python provides a set of list methods that makes it easy for us to store and retrieve values in order to maintain, update,
and extract data. 
\item These common operations are what Python programmers perform, including \textbf{slicing}, \textbf{sorting}, \textbf{appending}, \textbf{searching}, \textbf{inserting}, and \textbf{removing} data.
\end{itemize}

\end{frame}

\begin{frame}[fragile]{List Methods}
    \begin{columns}[t]
      \column{0.5\textwidth}
        \begin{exercise}{Basic List Operations}
      \begin{lstlisting}
        shopping = ["bread","milk", "eggs"]
        print(len(shopping))
      \end{lstlisting}
      \begin{lstlisting}
        list1 = [1,2,3]
        list2 = [4,5,6]
        final_list = list1 + list2
        print(final_list)
      \end{lstlisting}
   
      \begin{lstlisting}
        list3 = ['oi']
        print(list3*3)
      \end{lstlisting}
    \end{exercise}
      \column{0.5\textwidth}
      \begin{exercise}{Accessing an Item from Shopping List Data}
        \begin{lstlisting}
          shopping = ["bread","milk", "eggs"]
          print(shopping[1])

          shopping[1] = "banana"
          print(shopping)
          print(shopping[-1])

          print(shopping[0:2])
          print(shopping[:3])
          print(shopping[1:])
        \end{lstlisting}
      \end{exercise}
    \end{columns}

  \end{frame}

  \begin{frame}[fragile]{}
    \begin{exercise}{Adding Items to Our Shopping List}
      \begin{lstlisting}
        shopping = ["bread","milk", "eggs"]
        shopping.append("apple")
        print(shopping)

        shopping = []
        shopping.append('bread')
        shopping.append('milk')
        shopping.append('eggs')
        shopping.append('apple')
        print(shopping)
      \end{lstlisting}
      \begin{lstlisting}
        shopping.insert(2, 'ham')
        print(shopping)
      \end{lstlisting}
    \end{exercise}

    More on list methods in the following link \href{https://www.w3schools.com/python/python_ref_list.asp}{Python List/Array Methods}
  \end{frame}

  \section{Dictionary Keys and Values}
\label{sec:dictiories}

\begin{frame}[fragile]{Dictionary}
  \begin{onlyenv}<1>
      A Python dictionary is an unordered collection. Dictionaries are written with curly brackets, and they have \textbf{keys} and \textbf{values}.

  \begin{lstlisting}
    employee = {
      'name': "Jack Nelson",
      'age': 32,
      'department': "sales"
    }
  \end{lstlisting}
  \end{onlyenv}

  \begin{onlyenv}<2>
    \begin{exercise}{Using a Dictionary to Store a Movie Record}
      \begin{lstlisting}
        movie = {
          "title": "The Godfather",
          "director": "Francis Ford Coppola",
          "year": 1972,
          "rating": 9.2
        }
        
        print(movie['year'])
        movie['rating'] = (movie['rating'] + 9.3)/2
        print(movie['rating'])

        movie = {}
        movie['title'] = "The Godfather"
        movie['director'] = "Francis Ford Coppola"
        movie['year'] = 1972
        movie['rating'] = 9.2

        movie['actors'] = ['Marlon Brando', 'Al Pacino', 'James Caan']
        movie['other_details'] = {
          'runtime': 175,
          'language': 'English'
        }
        print(movie)
      \end{lstlisting}
  \end{exercise}
\end{onlyenv}
\end{frame}

\begin{frameact}{Storing Company Employee Table Data Using a List and a Dictionary}
  \begin{tabular}{lll}
    Name          & Age & Department \\
    \midrule
    John Mckee    & 38  & Sales      \\
    Lisa Crawford & 29  & Marketing  \\
    Sujan Patel   & 33  & HR        \\
    \bottomrule
  \end{tabular}

  \begin{enumerate} \justifying
  \item Create a list named \textbf{employees}.
  \item Create three dictionary objects inside employees to store the information of each employee.
  \item Print the employees variable.
  \item Print the details of all employees in a presentable format.
  \item Print only the details of Sujan Patel.
  \end{enumerate}
\end{frameact}

\begin{frame}[fragile]{Zipping and Unzipping Dictionaries Using \texttt{zip( )}}
  \begin{exercise}{}
    \begin{lstlisting}
      items = ['apple', 'orange', 'banana']
      quantity = [5, 3, 2]

      orders = zip(items,quantity)
      print(orders)

      orders = zip(items,quantity)
      print(list(orders))

      orders = zip(items,quantity)
      print(tuple(orders))

      orders = zip(items,quantity)
      print(dict(orders))
    \end{lstlisting}
  \end{exercise}
\end{frame}

\begin{frame}[fragile]{Dictionary Methods}
  \begin{exercise}{Accessing a Dictionary Using Dictionary Methods}
    \begin{lstlisting}
      orders = {'apple':5, 'orange':3, 'banana':2}
      print(orders.values())
      print(list(orders.values()))

      print(list(orders.keys()))

      for tuple in list(orders.items()):
          print(tuple)
    \end{lstlisting}
  \end{exercise}

  More on Dictionary Methods, click in the following link: \href{https://www.w3schools.com/python/python_ref_dictionary.asp}{Python Dictionary Methods}      
\end{frame}



%%% Local Variables:
%%% mode: latex
%%% TeX-master: "slides"
%%% End:

  \section{Tuples}
\label{sec:tuples}


\begin{frame}[fragile]{Tuples}
  \begin{onlyenv}<1>
    A tuple object is similar to a list, but it cannot be changed. Tuples are immutable sequences, which means their values cannot be changed after initialization. You use a tuple to represent fixed collections of items:
    \begin{lstlisting}
    weekdays_list = ['Monday', 'Tuesday', 'Wednesday', 'Thursday', 'Friday', 'Saturday', 'Sunday']
    
    weekdays_tuple = ('Monday', 'Tuesday', 'Wednesday','Thursday','Friday','Saturday', 'Sunday')
  \end{lstlisting}
  \end{onlyenv}

  \begin{onlyenv}<2>
    \begin{exercise}{Exploring Tuple Properties in Our Shopping List}
      \begin{lstlisting}
        t = ('bread', 'milk', 'eggs')
        print(len(t))

        t.append('apple')
        t[2] = 'apple'

        print(t + ('apple', 'orange'))
        print(t)

        t_mixed = 'apple', True, 3
        print(t_mixed)
        t_shopping = ('apple',3), ('orange',2), ('banana',5)
        print(t_shopping)
      \end{lstlisting}
    \end{exercise}
  \end{onlyenv}
\end{frame}


%%% Local Variables:
%%% mode: latex
%%% TeX-master: "slides"
%%% End:

  \section{Sets}
\label{sec:sets}

\begin{frame}[fragile]{Sets}
  \begin{onlyenv}<1>
      \begin{itemize} \parskip3mm \justifying
      \item Sets are a relatively new addition to the Python collection type. 
      \item They are unordered collections of unique and immutable objects that support operations mimicking mathematical set theory. 
      \item As sets do not allow multiple occurrences of the same element, they can be used to effectively prevent duplicate values.
  \end{itemize}
\end{onlyenv}

\begin{onlyenv}<2>
  \begin{exercise}{Using Sets in Python}
    \begin{lstlisting}
      s1 = set([1,2,3,4,5,6])
      print(s1)
      s2 = set([1,2,2,3,4,4,5,6,6])
      print(s2)
      s3 = set([3,4,5,6,6,6,1,1,2])
      print(s3)

      s4 = {"apple", "orange", "banana"}
      print(s4)

      s4.add('pineapple')
      print(s4)
    \end{lstlisting}
  \end{exercise}
\end{onlyenv}

\end{frame}

\begin{frame}[fragile]{Set Operations}
  \begin{exercise}{Implementing Set Operations}
    \begin{columns}[t]
      \column{0.5\textwidth}
          \begin{lstlisting}
      s5 = {1,2,3,4}
      s6 = {3,4,5,6}

      print(s5 | s6)
      print(s5.union(s6))

      print(s5 & s6)
      print(s5.intersection(s6))
      
      print(s5 - s6)
      print(s5.difference(s6))

      print(s5 <= s6)
      print(s5.issubset(s6))
    \end{lstlisting}
    \column{0.5\textwidth}
    \begin{lstlisting}
      s7 = {1,2,3}
      s8 = {1,2,3,4,5}
      
      print(s7 <= s8)
      print(s7.issubset(s8))

      print(s7 < s8)
      s9 = {1,2,3}
      s10 = {1,2,3}
      print(s9 < s10)
      print(s9 < s9)

      print(s8 >= s7)
      print(s8.issuperset(s7))
      print(s8 > s7)
      print(s8 > s8)
    \end{lstlisting}
      
    \end{columns}
  \end{exercise}
\end{frame}
%%% Local Variables:
%%% mode: latex
%%% TeX-master: "slides"
%%% End:

\end{document}



