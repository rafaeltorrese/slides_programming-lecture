\documentclass[../main.tex]{subfiles}
\graphicspath{{figs/}}



\title{Python Structures}
%\subtitle{OPERATIONS RESEARCH. \\ Mathematical Models} % 6 hours. 2 sessions
\AtBeginSection[] % Do nothing for \section* %
{
\begin{frame}<beamer> 
  \frametitle{Agenda}
  \tableofcontents[currentsection] 
\end{frame}
}

\begin{document}
% ==================================================
\begin{frame}
  \maketitle
\end{frame}
\begin{frame}{Agenda}
  \tableofcontents
\end{frame}
% ==================================================
\begin{frame}[fragile]{Introduction}
  \begin{itemize} \justifying
  \item   In programming languages, data structures refer to objects that can hold some data together, which means they are used to store a collection of related data.
  \item For instance, you can use a list to store our to-do items for the day. 
  \item The following is an example to show you how lists are coded:
    \begin{lstlisting}
      todo = ["pick up laundry", "buy Groceries", "pay electric bills"]
    \end{lstlisting}
  \item We can also use a dictionary object to store more complex information such as subscribers' details from our mailing list
    \begin{lstlisting}
      User = {
        "first_name": "Jack",
        "last_name":"White",
        "age": 41,
        "email": "jack.white@gmail.com"
     }
    \end{lstlisting}
  \item There are four types of data structures in Python: \alert{list}, \alert{tuple}, \alert{dictionary}, and \alert{set}.
  \end{itemize}
\end{frame}

\section{Lists}
\label{sec:lists}

\begin{frame}[fragile]{The Power of Lists}
  \begin{enumerate} \justifying
  \item A list is a type of container in Python that is used to store \alert{multiple data sets at the same time}. 
  \item Python lists are often compared to arrays in other programming languages, but they do a lot more.
  \item A list in Python is written within square brackets, \texttt{[ ]}. Each element in the list has its own distinct position and index. 
  \end{enumerate}

  \begin{exercise}{Working with Python Lists}
      \begin{lstlisting}
        shopping = ["bread", "milk", "eggs"]
        print(shopping)
        for item in shopping:
            print(item)                                                             
      \end{lstlisting}
  \end{exercise}
\end{frame}
\end{document}
