\section{Dictionary Keys and Values}
\label{sec:dictiories}

\begin{frame}[fragile]{Dictionary}
  \begin{onlyenv}<1>
      A Python dictionary is an unordered collection. Dictionaries are written with curly brackets, and they have \textbf{keys} and \textbf{values}.

  \begin{lstlisting}
    employee = {
      'name': "Jack Nelson",
      'age': 32,
      'department': "sales"
    }
  \end{lstlisting}
  \end{onlyenv}

  \begin{onlyenv}<2>
    \begin{exercise}{Using a Dictionary to Store a Movie Record}
      \begin{lstlisting}
        movie = {
          "title": "The Godfather",
          "director": "Francis Ford Coppola",
          "year": 1972,
          "rating": 9.2
        }
        
        print(movie['year'])
        movie['rating'] = (movie['rating'] + 9.3)/2
        print(movie['rating'])

        movie = {}
        movie['title'] = "The Godfather"
        movie['director'] = "Francis Ford Coppola"
        movie['year'] = 1972
        movie['rating'] = 9.2

        movie['actors'] = ['Marlon Brando', 'Al Pacino', 'James Caan']
        movie['other_details'] = {
          'runtime': 175,
          'language': 'English'
        }
        print(movie)
      \end{lstlisting}
  \end{exercise}
\end{onlyenv}
\end{frame}

\begin{frameact}{Storing Company Employee Table Data Using a List and a Dictionary}
  \begin{tabular}{lll}
    Name          & Age & Department \\
    \midrule
    John Mckee    & 38  & Sales      \\
    Lisa Crawford & 29  & Marketing  \\
    Sujan Patel   & 33  & HR        \\
    \bottomrule
  \end{tabular}

  \begin{enumerate} \justifying
  \item Create a list named \textbf{employees}.
  \item Create three dictionary objects inside employees to store the information of each employee.
  \item Print the employees variable.
  \item Print the details of all employees in a presentable format.
  \item Print only the details of Sujan Patel.
  \end{enumerate}
\end{frameact}

\begin{frame}[fragile]{Zipping and Unzipping Dictionaries Using \texttt{zip( )}}
  \begin{exercise}{}
    \begin{lstlisting}
      items = ['apple', 'orange', 'banana']
      quantity = [5, 3, 2]

      orders = zip(items,quantity)
      print(orders)

      orders = zip(items,quantity)
      print(list(orders))

      orders = zip(items,quantity)
      print(tuple(orders))

      orders = zip(items,quantity)
      print(dict(orders))
    \end{lstlisting}
  \end{exercise}
\end{frame}

\begin{frame}[fragile]{Dictionary Methods}
  \begin{exercise}{Accessing a Dictionary Using Dictionary Methods}
    \begin{lstlisting}
      orders = {'apple':5, 'orange':3, 'banana':2}
      print(orders.values())
      print(list(orders.values()))

      print(list(orders.keys()))

      for tuple in list(orders.items()):
          print(tuple)
    \end{lstlisting}
  \end{exercise}

  More on Dictionary Methods, click in the following link: \href{https://www.w3schools.com/python/python_ref_dictionary.asp}{Python Dictionary Methods}      
\end{frame}



%%% Local Variables:
%%% mode: latex
%%% TeX-master: "slides"
%%% End:
