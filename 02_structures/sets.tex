\section{Sets}
\label{sec:sets}

\begin{frame}[fragile]{Sets}
  \begin{onlyenv}<1>
      \begin{itemize} \parskip3mm \justifying
      \item Sets are a relatively new addition to the Python collection type. 
      \item They are unordered collections of unique and immutable objects that support operations mimicking mathematical set theory. 
      \item As sets do not allow multiple occurrences of the same element, they can be used to effectively prevent duplicate values.
  \end{itemize}
\end{onlyenv}

\begin{onlyenv}<2>
  \begin{exercise}{Using Sets in Python}
    \begin{lstlisting}
      s1 = set([1,2,3,4,5,6])
      print(s1)
      s2 = set([1,2,2,3,4,4,5,6,6])
      print(s2)
      s3 = set([3,4,5,6,6,6,1,1,2])
      print(s3)

      s4 = {"apple", "orange", "banana"}
      print(s4)

      s4.add('pineapple')
      print(s4)
    \end{lstlisting}
  \end{exercise}
\end{onlyenv}

\end{frame}

\begin{frame}[fragile]{Set Operations}
  \begin{exercise}{Implementing Set Operations}
    \begin{columns}[t]
      \column{0.5\textwidth}
          \begin{lstlisting}
      s5 = {1,2,3,4}
      s6 = {3,4,5,6}

      print(s5 | s6)
      print(s5.union(s6))

      print(s5 & s6)
      print(s5.intersection(s6))
      
      print(s5 - s6)
      print(s5.difference(s6))

      print(s5 <= s6)
      print(s5.issubset(s6))
    \end{lstlisting}
    \column{0.5\textwidth}
    \begin{lstlisting}
      s7 = {1,2,3}
      s8 = {1,2,3,4,5}
      
      print(s7 <= s8)
      print(s7.issubset(s8))

      print(s7 < s8)
      s9 = {1,2,3}
      s10 = {1,2,3}
      print(s9 < s10)
      print(s9 < s9)

      print(s8 >= s7)
      print(s8.issuperset(s7))
      print(s8 > s7)
      print(s8 > s8)
    \end{lstlisting}
      
    \end{columns}
  \end{exercise}
\end{frame}
%%% Local Variables:
%%% mode: latex
%%% TeX-master: "slides"
%%% End:
